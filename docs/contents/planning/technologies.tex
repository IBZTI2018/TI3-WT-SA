\subsection{Verwendete Technologien}

Zur Entwicklung werden, gemäss den für die Case-Study gemachten Vorgaben, folgende Technologien verwendet:

\begin{itemize}
  \item \emph{HTML5} und \emph{CSS3} für das Frontend, \emph{PHP7.3} für das Backend\footnote{Die Aufteilung in Frontend und Backend erfolgt hier nur theorethisch.}
  \item \emph{MariaDB} als SQL Datenbank, \emph{Adminer} als DB-Management Tool
  \item \emph{Nginx} Webserver (Nach Rücksprache mit dem Dozent, ersetzt \emph{Apache})
  \item \emph{Docker} und \emph{docker-compose} für die Entwicklungsumgebung
  \item \emph{VSCode} als Text-Editor
\end{itemize}

\noindent
Für das Deployment der einzelnen Produkt-Iterationen, sowie des fertigen Produkts, wird \emph{webtechdeploy}\footnote{Repository verfügbar unter \href{https://github.com/IBZTI2018/webtechdeploy}{github.com/IBZTI2018/webtechdeploy}} verwendet.
Dabei handelt es sich um ein ebenfalls selbst entwickeltes Projekt, mit welchem ein LEMP\footnote{LEMP ist ein Akronym für "Linux + nginx + MySQL + PHP"} Stack in einem isolierten Docker-Container via GitHub Webhook ausgeliefert werden kann.
Es wurde sich für diese Lösung entschieden, da im gegensatz zu "klassischen" Deployment Systemen der Ablauf bei diesem System stark vereinfacht ist, da es auf kleine Demo-Projekte wie dieses ausgelegt ist. So muss nur minimaler Aufwand in das Deployment gesteckt werden.
