\subsection{Entwicklungsumgebung}

Als Entwicklungsumgebung wird ein \emph{docker-compose} basierendes Containersystem verwendet. Als weitere Möglichkeiten standen \emph{XAMPP} oder eine manuelle Installation von PHP/MySQL zur Verfügung.

Es wurde sich für \emph{docker-compose} entschieden, da alle Teilnehmer auf verschiedenen Plattformen (OSX, Arch Linux, Windows 10) entwickeln und so die Installations- und Nutzungsvoraussetzungen auf jeder Plattform gleich sind. Zudem müssen so keine nativen Programme installiert werden, was es einfach macht, die Entwicklungsumgebung nach Abschluss des Projekts wieder zu entfernen.

Als Editor wird \emph{VSCode} verwendet, da die Plugin-Unterstützung für die jeweiligen Technologien hier sehr ausgeprägt ist. Eine vollumfängliche IDE wie \emph{PHPStorm} oder \emph{Netbeans} erschienen für den kleinen Umfang des Projekts als ein zu grosser Overhead.
