\subsection{Entwicklungsumgebung}

Als Entwicklungsumgebung wird ein \emph{docker-compose} basierendes Containersystem verwendet. Als weitere Möglichkeiten standen \emph{XAMPP} oder eine manuelle Installation von PHP/MySQL zur Verfügung.

Es wurde sich für Docker entschieden, da alle Teilnehmer auf verschiedenen Plattformen entwickeln und so die Installations- und Nutzungsvoraussetzungen auf jeder Plattform gleich sind.

Als Editor wird VSCode verwendet, da die Plugin-Unterstützung für die jeweiligen Technologien hier sehr ausgeprägt sind. Eine vollumfängliche IDE wie PHPStorm oder Netbeans erschienen für den kleinen Umfang des Projekts als ein zu grosser Overhead.
