\subsection{Milestones}
Bei der Einteilung der Funktionalen Anforderungen (FA) in soll und muss Kriterien, wurde vor allem die Aspekte Benutzerfreundlichkeit und Sicherheit betrachtet. So wurde es als wichtigstes Ziel erachtet, Nutzern sicher und voneinander isoliert Tagebücher zur Verfügung zu stellen.

Zudem wurden die Milestones so angeordnet, dass das Produkt möglichst früh in seiner Grundform\footnote{Nach dem \emph{Minimum viable product} - Prinzip} ohne manuelles Eingreifen (Datenbankmanipulation) benutzbar ist.

\emph{NFA\_1.3} und \emph{NFA\_1.4} sind in jedem Fall als gegeben anzunehmen, da es sich um eine Web-basierende Software handelt. \emph{NFA\_1.4} soll zu Projektende, falls es der Terminplan erlaubt, in einem Benchmarking belegt werden. Basierend auf den restlichen funktionalen Anforderungen (FA) und nicht-funktionalen Anforderungen (NFA) der Case-Study wurden folgende Milestones festgelegt:

\subsubsection*{M1 Benutzerlogin (muss)}
\textbf{Deadline: } 20.12.2019 (Ende Sprint 1). \\
\textbf{Anforderungen: }\emph{FA\_1.11}.\\
Benutzer sind in der Datenbank mit einem Benutzernamen und einem gehashten Passwort gespeichert. Ein Benutzer muss sich vor der Verwendung der Software einloggen. Eine Nutzung der Software ohne Login ist nicht möglich. Tagebücher sind einem Nutzer zugeordnet.

\subsubsection*{M2 Registrationssystem (muss)}
\textbf{Deadline: } 20.12.2019 (Ende Sprint 1). \\
\textbf{Anforderungen: }\emph{FA\_1.12}.\\
Ein neuer Nutzer soll sich mit einem Benutzernamen und einem selbst gewählten Passwort, ein Login erstellen können. Bei der Erstellung eines Logins wird für den Nutzer automatisch ein Tagebuch erstellt, in welchem er Einträge erfassen kann. Es ist nicht angedacht, dass ein Nutzer mehrere Tagebücher besitzen kann.

\subsubsection*{M3 Einträge erfassen (muss)}
\textbf{Deadline: } 03.01.2020 (Ende Sprint 2). \\
\textbf{Anforderungen: }\emph{FA\_1.1}, \emph{FA\_1.2}, \emph{FA\_1.4}, \emph{FA\_1.6} und \emph{FA\_1.13}.\\
Tagebucheinträge mit einem Datum und Text von bis zu 1000 Zeichen (Unicode) sollen erfasst werden können. Die Einträge werden in der Datenbank gespeichert und können wieder abgerufen werden. Das Bearbeiten oder Löschen von Einträgen ist nicht möglich. Jeder Benutzer kann nur Einträge in dem ihm zugeordneten Tagebuch sehen und erstellen.

\subsubsection*{M4 Einträge kategorisieren (muss)}
\textbf{Deadline: } 17.01.2020 (Ende Sprint 3). \\
\textbf{Anforderungen: }\emph{FA\_1.3} und \emph{FA\_1.7}.\\
Eintragskategorien können in der Datenbank erstellt werden, eine Verwaltungsoberfläche für die Verfügbaren Kategorien ist nicht Teil dieses Milestones. Tagebucheinträge können bei der Erstellung einer Kategorie zugeordnet werden, können aber auch ohne Kategorie erstellt werden.

\subsubsection*{M5 Filtern von Einträgen (soll)}
\textbf{Deadline: } 31.01.2020 (Ende Sprint 4). \\
\textbf{Anforderungen: }\emph{FA\_1.8}, \emph{FA\_1.9}.\\
Tagebucheinträge können nach einem bestimmten Datum oder einer von-bis Datums-Spanne gefiltert werden. Zudem können Einträge auch nach der zugeordneten Kategorie gefiltert werden.

\subsubsection*{M6 Einträge mit Bildern (soll)}
\textbf{Deadline: } 14.02.2020 (Ende Sprint 5). \\
\textbf{Anforderungen: }\emph{FA\_1.5}.\\
Bei der Erstellung eines Tagebucheintrags kann zusätzlich eine Bilddatei hochgeladen werden. Die Bilddatei wird auch beim Abrufen eines Tagebucheintrages wieder korrekt angezeigt.

\subsubsection*{M7 Optimierungen (soll)}
\textbf{Deadline: } 28.02.2020 (Ende Sprint 6). \\
\textbf{Anforderungen: }\emph{FA\_1.10}, \emph{NFA\_1.1} und \emph{NFA\_1.2}.\\
Zur Optimierung der Benutzbarkeit sollen Tage, an denen keine Einträge vorhanden sind, direkt angezeigt werden. Obwohl die bisherigen Milestones durch die Erarbeitung einer Funktionalität bereits die Erstellung einer intuitiven UI beinhalten, soll hier das Design noch einmal überarbeitet und wo möglich die Benutzerfreundlichkeit verbessert werden.
