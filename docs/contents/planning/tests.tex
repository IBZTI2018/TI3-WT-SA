\subsection{Testfälle}
Die Testfälle (Unit \& Functional tests) wurden mit PHP geschrieben und werden mit PHPUnit ausgeführt.\\
PHPUnit läuft auf eine separate Datenbank, bzw. bei einer Testdatenbank. \\
Sobald alle Testfälle ausgeführt worden sind, wird die ganze Datenbank entleert. \\
\textbf{NOTIZ: } Beim Aufsetzen gewisser Testfälle, wird die Testdatenbank mit Seed-Data befüllt.\\

\noindent
Nachfolgend werden sämtliche Testfälle aufgelistet.\\


\subsubsection*{\#1 Testabläufe für den Model: Category}
- Überprüfen ob bei nicht existierende Kategorie-ID, null zurückgegeben wird.\\
- Überprüfen ob bei existierende Kategorie-ID, Category zurückgegeben wird.

\subsubsection*{\#2 Testabläufe für die Datenbank}
- Überprüfen ob ein UNIQUE CONSTRAINT beim `username`-Feld aus der Tabelle `users` gesetzt ist.\\
- Überprüfen ob ein UNIQUE CONSTRAINT beim `category`-Feld aus der Tabelle `categories` gesetzt ist.

\subsubsection*{\#3 Testabläufe für den Model: Entry}
- Überprüfen ob die richtige Tageseinträge für ein bestimmten Benutzer aufgelistet werden.\\
- Überprüfen ob alle Tageseinträge vom Benutzer aus der Datenbank ausgelesen werden.\\
- Überprüfen ob FOREIGN KEY VIOLATION beim user\_id Feld aktiviert wird bei wiederholende IDs.\\
- Überprüfen ob beim user\_id existierende Foreign Key ID's akzeptiert werden.\\
- Überprüfen ob FOREIGN KEY VIOLATION beim category\_id Feld aktiviert wird bei wiederholende IDs.\\
- Überprüfen ob beim category\_id existierende Foreign Key ID's akzeptiert werden.\\
- Überprüfen ob das gespeicherte Bild, das richtige Encoding hat.\\
- Überprüfen ob Tageseinträge für ein bestimmtes Datenbereich aus der Datenbank ausgelesen werden.\\
- Überprüfen ob Tageseinträge für eine bestimmte Kategorie aus der Datenbank ausgelesen werden.\\
- Überprüfen ob Tageseinträge für ein bestimmtes Datenbereich sowie auch für eine bestimmte Kategorie aus der Datenbank ausgelesen werden.\\
- Überprüfen ob Tageseinträge fir eine nicht existierende Kategorie aus der Datenkbank nicht ausgelesen werden.\\
- Überprüfen ob bei angeklickte Checkbox, leere Tageseinträge für ein bestimmten Datumsbereich angezeigt werden, falls mindestens ein echten Eintrag vorhanden ist.\\
- Überprüfen ob bei angeklickte Checkbox, leere Tageseinträge für ein bestimmten Datumsbereich angezeigt werden, falls kein echten Eintrag vorhanden ist.\\
- Überprüfen ob bei nicht angeklickte Checkbox, keine leere Tageseinträge für ein bestimmten Datumsbereich angezeigt werden.\\
- Überprüfen ob bei angeklickte Checkbox, keine leere Tageseinträge für kein bestimmten Datumsbereich angezeigt werden.

\subsubsection*{\#4 Testabläufe für die 'Tagebucheintrag erstellen'-Seite}
- Überprüfen ob ein Fehler angezeigt wird, falls das Veröffentlichungsdatum leer ist.\\
- Überprüfen ob ein Fehler angezeigt wird, falls die Kategorie leer ist.\\
- Überprüfen ob ein Fehler angezeigt wird, falls der Inhalt leer ist.\\
- Überprüfen ob ein Fehler angezeigt wird, falls der Inhalt mehr als 1000 Zeichen beträgt.\\
- Überprüfen ob ein Tagebucheintrag erfolgreich erstellt werden kann.\\
- Überprüfen ob ein Tagebucheintrag mit ein Bild erfolgreich erstellt werden kann.

\subsubsection*{\#5 Testabläufe für die 'Einloggen'-Seite}
- Überprüfen ob ein Fehler angezeigt wird, falls der Benutzername leer ist.\\
- Überprüfen ob ein Fehler angezeigt wird, falls das Passwort leer ist.\\
- Überprüfen ob ein Fehler angezeigt wird, falls die Login-Daten falsch sind.\\
- Überprüfen ob man mit validen Login-Daten, sich einloggen kann.

\subsubsection*{\#6 Testabläufe für die 'Ausloggen'-Seite}
- Überprüfen ob nach dem Ausloggen, der Benutzer nicht mehr eingeloggt ist.

\subsubsection*{\#7 Testabläufe für die 'Tagebuch erstellen'-Seite}
- Überprüfen ob ein Fehler angezeigt wird, falls der Benutzername leer ist.\\
- Überprüfen ob ein Fehler angezeigt wird, falls das Passwort leer ist.\\
- Überprüfen ob ein Fehler angezeigt wird, falls die Passwörter nicht übereinstimmen.\\
- Überprüfen ob ein Fehler angezeigt wird, falls das angegebene Benutzername bereits verwendet wird.\\
- Überprüfen ob mit validen Daten ein Benutzer gelegt werden kann und ob nach der Registrierung, der Benutzer eingeloggt ist.

\subsubsection*{\#8 Testabläufe für das PBKDF2 Algorithmus}
- Überprüfen ob für das gleiche Passwort, zwei verschiedene Hashs generiert werden.

\subsubsection*{\#9 Testabläufe für den Model: User}
- Überprüfen ob das Einloggen bei nicht existierende Daten fehlschlägt.\\
- Überprüfen ob das Einloggen mit einem falschen Benutzername fehlschlägt.\\
- Überprüfen ob das Einloggen mit ein falsches Passwort fehlschlägt.\\
- Überprüfen ob das Einloggen erfolgreich ist.\\
- Überprüfen ob das Registrieren mit einem bereits existierender Benutzername fehlschlägt.\\
- Überprüfen ob das Registrieren mit validen Daten erfolgreicht ist.\\
- Überprüfen ob bei nicht existierende User-ID, null zurückgegeben wird.\\
- Überprüfen ob bei existierende User-ID, User zurückgegeben wird.\\
- Überprüfen ob bei nicht existierender Benutzername, null zurückgegeben wird.\\
- Überprüfen ob bei existierender Benutzename, User zurückgegeben wird.

\subsubsection*{\#10 Testabläufe für die 'Bildansicht'-Seite}
- Überprüfen ob ein Fehler angezeigt wird, falls entry\_id nicht definiert ist.\\
- Überprüfen ob ein Fehler angezeigt wird, falls entry\_id leer ist.\\
- Überprüfen ob ein Fehler angezeigt wird, falls für das angegebene entry\_id kein Objekt existiert.\\
- Überprüfen ob ein Fehler angezeigt wird, falls das Entry Objekt nicht zum eingeloggten Benutzer gehört.\\
- Überprüfen ob bei validen Daten, dass Bild angezeigt wird.