\subsection{Testfälle}
Die Testfälle (Unit \& Integrationtests) wurden mit PHP geschrieben und werden mit PHPUnit ausgeführt. PHPUnit greift dabei auf die selbe Datenbank zu, die Änderungen werden allerdings in Transactions isoliert, welche nicht in der Datenbank gespeichert werden.

Da die Tests komplett automatisiert mittels Test-Framework durchgeführt werden, wurde entschieden an dieser Stelle nicht die kompletten Testabläufe aufzuführen, sondern nur eine Liste von vorhandenen Testfällen zusammenzufassen.

\subsubsection*{Testabläufe für die Datenbank}
\begin{itemize}
  \item Überprüfen ob ein UNIQUE CONSTRAINT beim `username`-Feld aus der Tabelle `users` gesetzt ist.
  \item Überprüfen ob ein UNIQUE CONSTRAINT beim `category`-Feld aus der Tabelle `categories` gesetzt ist.
\end{itemize}

\subsubsection*{Testabläufe für das PBKDF2 Algorithmus}
\begin{itemize}
  \item Überprüfen ob für das gleiche Passwort zwei verschiedene Hashs generiert werden.
\end{itemize}

\subsubsection*{Testabläufe für das Category-Model}
\begin{itemize}
  \item Überprüfen ob bei nicht existierende Kategorie-ID, null zurückgegeben wird.
  \item Überprüfen ob bei existierende Kategorie-ID, Category zurückgegeben wird.
\end{itemize}

\subsubsection*{Testabläufe für das Entry-Model}
\begin{itemize}
  \item Überprüfen ob die richtigen Tagebucheinträge für ein bestimmten Benutzer aufgelistet werden.
  \item Überprüfen ob alle Tagebucheinträge vom Benutzer aus der Datenbank ausgelesen werden.
  \item Überprüfen ob FOREIGN KEY VIOLATION beim user\_id Feld aktiviert wird bei wiederholende IDs.
  \item Überprüfen ob beim user\_id existierende Foreign Key ID's akzeptiert werden.
  \item Überprüfen ob FOREIGN KEY VIOLATION beim category\_id Feld aktiviert wird bei wiederholende IDs.
  \item Überprüfen ob beim category\_id existierende Foreign Key ID's akzeptiert werden.
  \item Überprüfen ob das gespeicherte Bild, das richtige Encoding hat.
  \item Überprüfen ob Tagebucheinträge für einen bestimmten Datumsbereich aus der Datenbank ausgelesen werden.
  \item Überprüfen ob Tagebucheinträge für eine bestimmte Kategorie aus der Datenbank ausgelesen werden.
  \item Überprüfen ob Tagebucheinträge für ein bestimmtes Datumsbereich sowie auch für eine bestimmte Kategorie aus der Datenbank ausgelesen werden.
  \item Überprüfen ob Tagebucheinträge für eine nicht existierende Kategorie aus der Datenkbank nicht ausgelesen werden.
  \item Überprüfen ob bei angeklickter Checkbox leere Tagebucheinträge für ein bestimmten Datumsbereich angezeigt werden, falls mindestens ein echter Eintrag vorhanden ist.
  \item Überprüfen ob bei angeklickter Checkbox, leere Tagebucheinträge für ein bestimmten Datumsbereich angezeigt werden, falls kein echter Eintrag vorhanden ist.
  \item Überprüfen ob bei nicht angeklickter Checkbox, keine leeren Tagebucheinträge für ein bestimmten Datumsbereich angezeigt werden.
  \item Überprüfen ob bei angeklickter Checkbox, keine leeren Tagebucheinträge für keinen bestimmten Datumsbereich angezeigt werden.
\end{itemize}

\subsubsection*{Testabläufe für das User-Model}
\begin{itemize}
  \item Überprüfen ob das Einloggen bei nicht existierende Daten fehlschlägt.
  \item Überprüfen ob das Einloggen mit einem falschen Benutzername fehlschlägt.
  \item Überprüfen ob das Einloggen mit ein falsches Passwort fehlschlägt.
  \item Überprüfen ob das Einloggen erfolgreich ist.
  \item Überprüfen ob das Registrieren mit einem bereits existierender Benutzername fehlschlägt.
  \item Überprüfen ob das Registrieren mit validen Daten erfolgreicht ist.
  \item Überprüfen ob bei nicht existierende User-ID, null zurückgegeben wird.
  \item Überprüfen ob bei existierende User-ID, User zurückgegeben wird.
  \item Überprüfen ob bei nicht existierender Benutzername, null zurückgegeben wird.
  \item Überprüfen ob bei existierender Benutzename, User zurückgegeben wird.
\end{itemize}

\subsubsection*{Testabläufe für die 'Tagebucheintrag erstellen'-Seite}
\begin{itemize}
  \item Überprüfen ob ein Fehler angezeigt wird, falls das Veröffentlichungsdatum leer ist.
  \item Überprüfen ob ein Fehler angezeigt wird, falls die Kategorie leer ist.
  \item Überprüfen ob ein Fehler angezeigt wird, falls der Inhalt leer ist.
  \item Überprüfen ob ein Fehler angezeigt wird, falls der Inhalt mehr als 1000 Zeichen beträgt.
  \item Überprüfen ob ein Tagebucheintrag erfolgreich erstellt werden kann.
  \item Überprüfen ob ein Tagebucheintrag mit ein Bild erfolgreich erstellt werden kann.
\end{itemize}

\subsubsection*{Testabläufe für die 'Einloggen'-Seite}
\begin{itemize}
  \item Überprüfen ob ein Fehler angezeigt wird, falls der Benutzername leer ist.
  \item Überprüfen ob ein Fehler angezeigt wird, falls das Passwort leer ist.
  \item Überprüfen ob ein Fehler angezeigt wird, falls die Login-Daten falsch sind.
  \item Überprüfen ob man mit validen Login-Daten, sich einloggen kann.
\end{itemize}

\subsubsection*{Testabläufe für die 'Ausloggen'-Seite}
\begin{itemize}
  \item Überprüfen ob nach dem Ausloggen der Benutzer nicht mehr eingeloggt ist.
\end{itemize}

\subsubsection*{Testabläufe für die 'Tagebuch erstellen'-Seite}
\begin{itemize}
  \item Überprüfen ob ein Fehler angezeigt wird, falls der Benutzername leer ist.
  \item Überprüfen ob ein Fehler angezeigt wird, falls das Passwort leer ist.
  \item Überprüfen ob ein Fehler angezeigt wird, falls die Passwörter nicht übereinstimmen.
  \item Überprüfen ob ein Fehler angezeigt wird, falls das angegebene Benutzername bereits verwendet wird.
  \item Überprüfen ob mit validen Daten ein Benutzer gelegt werden kann und ob nach der Registrierung, der Benutzer eingeloggt ist.
\end{itemize}

\subsubsection*{Testabläufe für die 'Bildansicht'-Seite}
\begin{itemize}
  \item Überprüfen ob ein Fehler angezeigt wird, falls entry\_id nicht definiert ist.
  \item Überprüfen ob ein Fehler angezeigt wird, falls entry\_id leer ist.
  \item Überprüfen ob ein Fehler angezeigt wird, falls für das angegebene entry\_id kein Objekt existiert.
  \item Überprüfen ob ein Fehler angezeigt wird, falls das Entry Objekt nicht zum eingeloggten Benutzer gehört.
  \item Überprüfen ob bei validen Daten, dass Bild angezeigt wird.
\end{itemize}
