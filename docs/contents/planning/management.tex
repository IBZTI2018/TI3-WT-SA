\subsection{Projektmanagement}
Für die Planung und Durchführung der Case-Study wird eine agile Vorgehensweise nach SCRUM gewählt. Der Sprint-Zyklus wird hierbei auf \emph{2 Wochen} festgelegt. Ein Sprint an sich dauert 12 Tage, zudem befinden sich zwischen zwei Sprints jeweils 2 freie Tage, an denen Sprint-Review und Sprint-Retrospective durchgeführt werden.

Bedingt durch die fixen Termine der Case-Study, wurde die Arbeitszeit in \emph{6 Sprints} unterteilt. Damit die Sprints als konstante Zeiteinheit repräsentativ für Schätzungen sind, obwohl die Case-Study nebst Studium und Beruf durchgeführt wird, wird jedem Sprint ein fixes Zeitbudget von \emph{13 Lektionen pro Teilnehmer}\footnote{6x13h = 78h pro Teilnehmer, exklusive Planungsarbeit vor dem ersten Sprint} zugewiesen.

Analysen und Dokumentationen werden als Bestandteil jeder User-Story laufend ergänzt und in der Zeitschätzung miteingerechnet. Die nachfolgende Tabelle listet Sprints, wie sie ursprünglich geplant waren, sowie wie sie tatsächlich durchgeführt wurden.

\begin{center}
  \begin{tabular}{ l l l }
    \textbf{Sprint} & \textbf{Geplant} & \textbf{Durchgeführt} \\
    Sprint 1 & 09.12.2019 - 20.12.2019 & 09.12.2019 - 20.12.2019 \\
    Sprint 2 & 23.12.2019 - 03.01.2020 & - \\
    Sprint 3 & 06.01.2020 - 17.01.2020 & 06.01.2020 - 17.01.2020 \\
    Sprint 4 & 20.01.2020 - 31.01.2020 & 21.01.2020 - 31.01.2020 \\
    Sprint 5 & 03.02.2020 - 14.02.2020 & 10.02.2020 - 21.02.2020 \\
    Sprint 6 & 17.02.2020 - 28.02.2020 & - \\
    \textcolor{white}{.......................} &&\\
  \end{tabular}
\end{center}

\noindent
Wie in der Liste ersichtlich, konnte die ursprünglich geplante Sprintverteilung nicht eingehalten werden. Besonders über die Feiertage, sowie gegen Semesterende wurden Sprints ausgelassen oder umgeplant, da die Teilnehmer in dieser Zeit Privat oder Studiumsbedingt sehr ausgelastet waren.

Es hat sich gezeigt, dass eine regelmässige Sprintplanung nur möglich ist, wenn jeder Mitarbeitende ein fixes Zeitbudget in der entsprechenden Zeit zur verfügung stellen kann. Für Arbeiten während des Studiums erscheint dieser Ansatz also eher umständlich.

Trotz relativ grosser Änderungen in der Sprintplanung konnten alle Milestones erfüllt werden. Dies liegt vor allem daran, dass die Zeit einzelner User-Stories stets nur auf die Umsetzungszeit geschätzt wurde, die Zeit für Review und Feedback allerdings nicht mit eingerechnet wurde.
