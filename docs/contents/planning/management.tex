\subsection{Projektmanagement}
Für die Planung und Durchführung der Case-Study wird eine agile Vorgehensweise nach SCRUM gewählt. Der Sprint-Zyklus wird hierbei auf \emph{2 Wochen} festgelegt. Ein Sprint an sich dauert 12 Tage, zudem befinden sich zwischen zwei Sprints jeweils 2 freie Tage, an denen Sprint-Review und Sprint-Retrospect durchgeführt werden.

Bedingt durch die fixen Termine der Case-Study, wurde die Arbeitszeit in \emph{6 Sprints} unterteilt. Damit die Sprints als konstante Zeiteinheit repräsentativ für Schätzungen sind, obwohl die Case-Study nebst Studium und Beruf durchgeführt wird, wird jedem Sprint ein fixes Zeitbudget von \emph{13 Lektionen pro Teilnehmer}\footnote{6x13h = 78h pro Teilnehmer, exklusive Planungsarbeit vor dem ersten Sprint} zugewiesen.

Analysen und Dokumentationen werden als Bestandteil jeder User-Story laufend ergänzt und in der Zeitschätzung miteingerechnet. Unter Berücksichtigung des 29.02.2020 als Abgabetermin, werden als Sprintanfang und -ende voraussichtlich folgende Termine verwendet:

\begin{center}
  \begin{tabular}{ l l } 
    09.12.2019 - 20.12.2019 & Sprint 1 (M1, M2) \\ 
    23.12.2019 - 03.01.2020 & Sprint 2 (M3) \\ 
    06.01.2020 - 17.01.2020 & Sprint 3 (M4) \\ 
    20.01.2020 - 31.01.2020 & Sprint 4 (M5) \\
    03.02.2020 - 14.02.2020 & Sprint 5 (M6) \\
    17.02.2020 - 28.02.2020 & Sprint 6 (M7) \\ 
  \end{tabular}
\end{center}

\noindent
Wie in der Liste ersichtlich, sind die Milestones des Projektes relativ gleichmässig auf alle Sprints verteilt. Somit hat jeder Sprint ein klares Sprintziel. Die User-Stories sollten vor Sprintbeginn in Rohfassung vorhanden sein und werden zum Sprintanfang \emph{gegroomt}. Die Dokumentation zu erweitern ist Bestandteil jedes Milestones.
