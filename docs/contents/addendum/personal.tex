\subsection{Persönliche Erkenntnisse}
\subsubsection*{Findung Lösungsvariante}
Dank vorhandener Vorerfahrung, war es für eine technisch simple Aufgabe wie dieses Projekt kein grosses Problem, eine relativ optimale Lösung zu finden.

Nach dem Abschluss des Projektmanagement-Workshops steht fest, dass die Voranalyse und Entscheidungsfindung definitiv professioneller und strukturierter hätte ausfallen können.

Es wurde entschieden, diese nicht rückwirkend nachzuarbeiten, da bereits ein nicht unwesentlicher Teil des Projekts implementiert wurde. Stattdessen werden relevante technische Entscheidung in schriftlicher Form im Verlauf dieses Dokuments festgehalten.

\subsubsection*{Sprintplanung Festtage}
Die ursprüngliche Sprintplanung rechnete die Festtage 2019/2020 mit ein.
Es war zwar bewusst, dass dies unter Umständen nicht korrekt funktionieren würde, es wurde aber entschieden, diesen Ansatz trotzdem zu probieren.
Rückblickend, wäre es sinnvoller gewesen, die Festtage grosszügig zu umplanen und dafür den vorherigen und nachfolgenden Sprint ein bisschen dichter zu packen.

\subsection*{Code-Review}
Es wurde entschieden sämtliche Features als eigene Feature-Branches zu entwickeln. Vor dem Merge eines Feature-Branches muss eine Code-Review vom jeweils anderen Teilnehmer durchgeführt werden.

Hierfür wurde relativ viel Zeit eingeplant, das Code-Reviews tendentiell aufwändig sind, vor allem wenn Revisionen notwendig sind. Durch die Einfachheit des Projekts wurden Code-Reviews immer viel schneller als erwartet abgeschlossen.

Die Zeitplanung wird für zukünftige Sprints allerdings nicht angepasst. Code-Reviews mehr als genug Zeit zuzuweisen erscheint sinnvoll, sollte ein Feature tatsächlich schwerwiegende Fehler aufweisen.

Zeit, welche an dieser Stelle gespart wird, kann zu Projektabschluss immer noch in Optimierungen und Verbesserung der Dokumentation investiert werden.
