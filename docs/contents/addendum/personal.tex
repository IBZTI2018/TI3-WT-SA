\subsection{Persönliche Erkenntnisse}
\subsubsection*{Vorgehensmodell Agil}
Das agile Vorgehensmodell hat sich bei uns während unserer Zusammenarbeit sehr bewährt. Gerade durch die vorhandene Vorerfahrung beider Teilnehmer, war so ein sehr produktives Arbeiten möglich.

Beim Zusammenstellen der Unterlagen für die Abgabe und der erneuten Einsicht der Bewertungskriterien kam mit dem Rückblick auf den mittlerweile stattgefundenen Projektmanagement-Workshop allerdings ein böses Erwachen.

Das Bewertungsschema der Case-Study ist fast Ausnahmslos auf Projektorganisation nach klassischem Wasserfallmodell ausgelegt. Wir waren uns dessen zwar bewusst, haben aber erst während dieses Projekte gemertkt, wie weitreichend die Konsequenzen tatsächlich sind.

Als eigenes Fazit daraus wollen wir zukünftige Case-Studies an der IBZ nicht mehr nach dem agilen Modell durchführen. Trotzdem war dieses Projekt eine sehr interessante Lernerfahrung, da es uns gezeigt hat, welche Vorabklärungen auch bei agiler Entwicklung notwendig sind, was uns auch im beruflichen Umfeld bereits mehrfach negativ aufgefallen ist.

Die Lösungsfindung in dieser Dokumentation wurde nachträglich erfasst, um unsere Gedankengänge bestmöglich ''nach Lehrbuch'' niederzuschreiben. Für uns persönlich hat sich damit aber gezeigt, dass ein Abwägen von Lösungsvarianten auch bei agiler Entwicklung zumindest oberflächlich stattfinden sollte, um sehr grosse Rewrites und Rückschritte zu verhindern.

Wir mögen so also ein paar Punkte nach dem Bewertungsschema dieser Case-Study verloren haben, haben aber etwas sehr wertvolles für den Berufsalltag mitgenommen.

\subsubsection*{Sprintplanung Festtage}
Die ursprüngliche Sprintplanung rechnete die Festtage 2019/2020 mit ein.
Es war zwar bewusst, dass dies unter Umständen nicht korrekt funktionieren würde, es wurde aber entschieden, diesen Ansatz trotzdem zu probieren.
Rückblickend, wäre es sinnvoller gewesen, die Festtage grosszügig zu umplanen und dafür den vorherigen und nachfolgenden Sprint ein bisschen dichter zu packen.

\subsubsection*{Code-Review}
Es wurde entschieden sämtliche Features als eigene Feature-Branches zu entwickeln. Vor dem Merge eines Feature-Branches muss eine Code-Review vom jeweils anderen Teilnehmer durchgeführt werden.

Hierfür wurde relativ viel Zeit eingeplant, das Code-Reviews tendentiell aufwändig sind, vor allem wenn Revisionen notwendig sind. Durch die Einfachheit des Projekts wurden Code-Reviews immer viel schneller als erwartet abgeschlossen.

Die Zeitplanung wird für zukünftige Sprints allerdings nicht angepasst. Code-Reviews mehr als genug Zeit zuzuweisen erscheint sinnvoll, sollte ein Feature tatsächlich schwerwiegende Fehler aufweisen.

Zeit, welche an dieser Stelle gespart wird, kann zu Projektabschluss immer noch in Optimierungen und Verbesserung der Dokumentation investiert werden.
