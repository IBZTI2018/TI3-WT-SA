\subsection{Zeitabrechnung}
Das Aufsetzen und schreiben der ersten Absätze der Dokumentation wurde vorab nicht geschätzt. Es wurde stattdessen im Rahmen einer kleinen Machbarkeits-Studie durchgeführt, um zu evaluieren, ob unser selbst gestecktes Ziel - das Schreiben der Dokumentation in \LaTeX - realistisch sei.

Nachfolgend werden sämtliche Tätigkeiten des Projekts mitsamt geplanter, effektiver und Delta-Zeit im Format \emph{h:mm} aufgelistet. Für den Entwicklungsverlauf wird statt einer Beschreibung die User-Story ID im Format \emph{\#id} angegeben. Die entsprechenden User-Storys finden sich im vorherigen Abschnitt.

\vspace{5mm}

\begin{longtable}{ c c c l l }
  \textbf{Plan} & \textbf{Eff.} & \textbf{Delta} & \textbf{Teiln.} & \textbf{Aufgabe} \\
  - & 2:00 & +2:00 & Sven & LaTeX Dokumentation Einrichten \\
  - & 2:00 & +2:00 & Beide & Dokumentation Organisationsform, Technik, VC\\
  1:00 & 0:40 & -0:20 & Beide & Stakeholder-Analyse, Milestone-Definition (Webmeeting) \\
  2:00 & 3:00 & +1:00 & Sven & Docker, docker-compose, PHPUnit setup \\
  - & 0:30 & +0:30 & Sven & Dokumentations-Abgabe Zwischentermin \\ 
  0:30 & 0:40 & +0:10 & Beide & Grooming Sprint 1 (Webmeeting) \\
  3:00 & 6:00 & +3:00 & André & \#2 Nutzerdaten in der Datenbank speichern \\
  3:00 & 3:00 & +0:00 & André & \#3 Anmelden mit Benutzerdaten \\
  2:00 & 3:00 & +1:00 & Sven & \#4 Registrieren mit Benutzerdaten \\
  1:00 & 1:30 & +0:30 & André & \#5 Passwörter mit pbkdf2 hashen \\
  5:00 & 3:30 & -1:30 & Beide & Code-Review vor Merging \\
  0:30 & 1:00 & +0:30 & Beide & Grooming Sprint 2 (Webmeeting) \\
  1:00 & 2:00 & +1:00 & Sven & \#10 Datenbank-Hook für Deployment \\
  3:00 & 3:00 & +0:00 & André & \#11 Neuen Tagebucheintrag erfassen \\
  2:00 & 1:30 & -0:30 & André & \#12 Kategorien erstellen \\
  1:00 & 0:30 & -0:30 & Sven & \#13 Ausloggen unterstützen \\
  2:00 & 4:00 & +2:00 & Sven & \#14 Grundlegendes Design \\
  3:00 & 1:30 & -1:30 & Sven & \#15 Einträge anzeigen \\
  5:00 & 2:30 & -2:30 & Beide & Code-Review vor Merging \\
  4:00 & 4:30 & +0:30 & Sven & Dokumentation bereinigen \\
  1:00 & 1:00 & +0:00 & André & #22 Commit-Version anzeigen \\
  0:15 & 0:15 & +0:00 & André & Call-to-action Knopf und Umbennenung \\
  3:00 & 3:00 & +0:00 & André & Eintrage filtern \\
  %\hline
  %29:00 & 36:20 & +7:20 & & von 2*60h gesamt. \\
\end{longtable}
