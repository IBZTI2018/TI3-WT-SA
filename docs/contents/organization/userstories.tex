\subsection{User-Stories}
Nachfolgend werden sämtliche bearbeiteten User-Stories mit ihren Acceptance-Criterias aufgelistet.


\subsubsection*{\#2 Nutzerdaten in der Datenbank speichern}
Als Systemadministrator will ich Nutzerdaten in der Datenbank speichern können, um ein Verzeichnis von Tagebuchnutzern anzulegen. Dafür soll für jeden Nutzer mindestens der Benutzername, sowie das Passwort gespeichert werden.
  
\noindent
- Als Systemadministartor kann ich die Datenbank mit einem Setup-Script aufsetzen.\\
- Als Systemadministrator kann ich einen Benutzer in der Datenbank via adminer hinzufügen.\\
- Es kann kein Benutzername doppelt benutzt werden.


\subsubsection*{\#3 Anmelden mit Benutzerdaten}
Als Endnutzer möchte ich mich über einen Login-Screen anmelden können, um die Applikation zu nutzen und ein tolles Tagebuch zu schreiben.

\noindent
- Beim Aufrufen der Applikation erscheint ein Login-Bildschirm.\\
- Als Benutzer kann ich mich mit meinen Benutzername und Passwort einloggen.\\
- Als Benutzer möchte ich eine informative Fehlermeldung erhalten, wenn ich falsche Zugangsdaten eingebe.\\
- Nach erfolgreichem Login möchte ich weiter geleitet werden.


\subsubsection*{\#4 Registrieren mit neuen Benutzerdaten}
Als neuer Enduser möchte ich mich registrieren können um mit den bereits bestehenden Benutzern zusammen den Dienst nutzen zu können.

\noindent
- Als Nutzer kann ich auf der Login-Seite einen neuen Account erstellen.\\
- Für das Erstellen eines neuen Accounts muss ich einen Benutzernamen und ein Passwort eingeben.\\
- Zum Schutz vor Vertippen muss das Passwort zwei mal eingegeben werden.\\
- Stimmen die Passwörter nicht überein, oder ist ein Feld leer, erscheint eine Fehlermeldung.\\
- Nach dem Registrieren werde ich wie ein eingeloggter Benutzer auf die interne Seite weitergeleitet.


\subsubsection*{\#5 Passwörter mit pbkdf2 hashen}
Als Systemadministrator will ich, dass Passwörter in der Datenbank nur gehasht gespeichert werden, um unsere Sicherheitsstandards zu erfüllen.

\noindent
- Passwörter sind in der Datenbank nur in pbkdf2 gehashter Form gespeichert.\\
- Login und Registrierung funktionieren weiterhin.


\subsubsection*{\#10 Datenbank-Reset Hook für Deployment}
Als Systemadministrator möchte ich, dass die Datenbank bei einem Deployment zurückgesetzt und mit neuen Demo-Daten geseedet wird, damit immer eine Demo-Version des Projekts verfügbar ist.

\noindent
- Datenbank wird bei deployment zurückgesetzt und geseedet


\subsubsection*{\#11 Neuen Tagebucheintrag erfassen}
Als Benutzer möchte ich neue Tagebucheinträge erstellen können, um meine Erlebnisse festzuhalten. Ein Eintrag soll dabei ein Datum, einen Text (bis 1000 Zeichen) und eine Kategorie enthalten.

\noindent
- Nutzer kann nach dem Einloggen einen Link anklicken um einen neuen Eintrag zu erstellen.\\
- Nutzer kann ein Datum auswählen, an dem der Eintrag erstellt werden soll.\\
- Nutzer kann einen Text eingeben, der gespeichert werden soll.\\
- Nutzer kann eine Kategorie auswählen, zu der der Eintrag gehört.\\
- Nutzer kann nur seine selbst erstellten Einträge sehen.


\subsubsection*{\#12 Kategorien erstellen}
Als Benutzer möchte ich meinen Tagebucheintrag einer Kategorie zuordnen können, um Ordnung in meinem Tagebuch zu halten.

\noindent
- Kategorie-Tabelle existiert und hat Standard-Kategorien


\subsubsection*{\#13 Ausloggen unterstützen}
Als Benutzer möchte ich mich ausloggen können, um mein Tagebuch vor neugierigen Menschen zu schützen.

\noindent
- Eingeloggter Nutzer kann sich ausloggen
- Nach dem Ausloggen ist ein Nutzer nicht mehr eingeloggt


\subsubsection*{\#14 Grundlegendes Design}
Als Benutzer möchte ich, dass das Tagebuch ein halbwegs schönes und zeitgemässes Design besitzt, damit es angenehm zu verwenden ist. (Timebox: 2h)

\noindent
- Tagebuch muss ein grundlegendes Grid-Design besitzen


\subsubsection*{\#15 Anzeigen von Einträgen}
Als Benutzer möchte ich meine bestehenden Einträge anzeigen können, damit ich mein Tagebuch lesen kann.

\noindent
- Benutzer kann seine erstellten Einträge einsehen\\
- Benutzer kann seine erstellten Einträge nicht bearbeiten\\
- Benutzer kann nur seine eigenen Einträge sehen
